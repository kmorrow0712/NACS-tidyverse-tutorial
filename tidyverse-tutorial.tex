% Options for packages loaded elsewhere
\PassOptionsToPackage{unicode}{hyperref}
\PassOptionsToPackage{hyphens}{url}
%
\documentclass[
]{article}
\usepackage{amsmath,amssymb}
\usepackage{lmodern}
\usepackage{ifxetex,ifluatex}
\ifnum 0\ifxetex 1\fi\ifluatex 1\fi=0 % if pdftex
  \usepackage[T1]{fontenc}
  \usepackage[utf8]{inputenc}
  \usepackage{textcomp} % provide euro and other symbols
\else % if luatex or xetex
  \usepackage{unicode-math}
  \defaultfontfeatures{Scale=MatchLowercase}
  \defaultfontfeatures[\rmfamily]{Ligatures=TeX,Scale=1}
\fi
% Use upquote if available, for straight quotes in verbatim environments
\IfFileExists{upquote.sty}{\usepackage{upquote}}{}
\IfFileExists{microtype.sty}{% use microtype if available
  \usepackage[]{microtype}
  \UseMicrotypeSet[protrusion]{basicmath} % disable protrusion for tt fonts
}{}
\makeatletter
\@ifundefined{KOMAClassName}{% if non-KOMA class
  \IfFileExists{parskip.sty}{%
    \usepackage{parskip}
  }{% else
    \setlength{\parindent}{0pt}
    \setlength{\parskip}{6pt plus 2pt minus 1pt}}
}{% if KOMA class
  \KOMAoptions{parskip=half}}
\makeatother
\usepackage{xcolor}
\IfFileExists{xurl.sty}{\usepackage{xurl}}{} % add URL line breaks if available
\IfFileExists{bookmark.sty}{\usepackage{bookmark}}{\usepackage{hyperref}}
\hypersetup{
  pdftitle={R Notebook},
  pdfauthor={Kelly Morrow (kmorrow@umd.edu)},
  hidelinks,
  pdfcreator={LaTeX via pandoc}}
\urlstyle{same} % disable monospaced font for URLs
\usepackage[margin=1in]{geometry}
\usepackage{color}
\usepackage{fancyvrb}
\newcommand{\VerbBar}{|}
\newcommand{\VERB}{\Verb[commandchars=\\\{\}]}
\DefineVerbatimEnvironment{Highlighting}{Verbatim}{commandchars=\\\{\}}
% Add ',fontsize=\small' for more characters per line
\usepackage{framed}
\definecolor{shadecolor}{RGB}{248,248,248}
\newenvironment{Shaded}{\begin{snugshade}}{\end{snugshade}}
\newcommand{\AlertTok}[1]{\textcolor[rgb]{0.94,0.16,0.16}{#1}}
\newcommand{\AnnotationTok}[1]{\textcolor[rgb]{0.56,0.35,0.01}{\textbf{\textit{#1}}}}
\newcommand{\AttributeTok}[1]{\textcolor[rgb]{0.77,0.63,0.00}{#1}}
\newcommand{\BaseNTok}[1]{\textcolor[rgb]{0.00,0.00,0.81}{#1}}
\newcommand{\BuiltInTok}[1]{#1}
\newcommand{\CharTok}[1]{\textcolor[rgb]{0.31,0.60,0.02}{#1}}
\newcommand{\CommentTok}[1]{\textcolor[rgb]{0.56,0.35,0.01}{\textit{#1}}}
\newcommand{\CommentVarTok}[1]{\textcolor[rgb]{0.56,0.35,0.01}{\textbf{\textit{#1}}}}
\newcommand{\ConstantTok}[1]{\textcolor[rgb]{0.00,0.00,0.00}{#1}}
\newcommand{\ControlFlowTok}[1]{\textcolor[rgb]{0.13,0.29,0.53}{\textbf{#1}}}
\newcommand{\DataTypeTok}[1]{\textcolor[rgb]{0.13,0.29,0.53}{#1}}
\newcommand{\DecValTok}[1]{\textcolor[rgb]{0.00,0.00,0.81}{#1}}
\newcommand{\DocumentationTok}[1]{\textcolor[rgb]{0.56,0.35,0.01}{\textbf{\textit{#1}}}}
\newcommand{\ErrorTok}[1]{\textcolor[rgb]{0.64,0.00,0.00}{\textbf{#1}}}
\newcommand{\ExtensionTok}[1]{#1}
\newcommand{\FloatTok}[1]{\textcolor[rgb]{0.00,0.00,0.81}{#1}}
\newcommand{\FunctionTok}[1]{\textcolor[rgb]{0.00,0.00,0.00}{#1}}
\newcommand{\ImportTok}[1]{#1}
\newcommand{\InformationTok}[1]{\textcolor[rgb]{0.56,0.35,0.01}{\textbf{\textit{#1}}}}
\newcommand{\KeywordTok}[1]{\textcolor[rgb]{0.13,0.29,0.53}{\textbf{#1}}}
\newcommand{\NormalTok}[1]{#1}
\newcommand{\OperatorTok}[1]{\textcolor[rgb]{0.81,0.36,0.00}{\textbf{#1}}}
\newcommand{\OtherTok}[1]{\textcolor[rgb]{0.56,0.35,0.01}{#1}}
\newcommand{\PreprocessorTok}[1]{\textcolor[rgb]{0.56,0.35,0.01}{\textit{#1}}}
\newcommand{\RegionMarkerTok}[1]{#1}
\newcommand{\SpecialCharTok}[1]{\textcolor[rgb]{0.00,0.00,0.00}{#1}}
\newcommand{\SpecialStringTok}[1]{\textcolor[rgb]{0.31,0.60,0.02}{#1}}
\newcommand{\StringTok}[1]{\textcolor[rgb]{0.31,0.60,0.02}{#1}}
\newcommand{\VariableTok}[1]{\textcolor[rgb]{0.00,0.00,0.00}{#1}}
\newcommand{\VerbatimStringTok}[1]{\textcolor[rgb]{0.31,0.60,0.02}{#1}}
\newcommand{\WarningTok}[1]{\textcolor[rgb]{0.56,0.35,0.01}{\textbf{\textit{#1}}}}
\usepackage{longtable,booktabs,array}
\usepackage{calc} % for calculating minipage widths
% Correct order of tables after \paragraph or \subparagraph
\usepackage{etoolbox}
\makeatletter
\patchcmd\longtable{\par}{\if@noskipsec\mbox{}\fi\par}{}{}
\makeatother
% Allow footnotes in longtable head/foot
\IfFileExists{footnotehyper.sty}{\usepackage{footnotehyper}}{\usepackage{footnote}}
\makesavenoteenv{longtable}
\usepackage{graphicx}
\makeatletter
\def\maxwidth{\ifdim\Gin@nat@width>\linewidth\linewidth\else\Gin@nat@width\fi}
\def\maxheight{\ifdim\Gin@nat@height>\textheight\textheight\else\Gin@nat@height\fi}
\makeatother
% Scale images if necessary, so that they will not overflow the page
% margins by default, and it is still possible to overwrite the defaults
% using explicit options in \includegraphics[width, height, ...]{}
\setkeys{Gin}{width=\maxwidth,height=\maxheight,keepaspectratio}
% Set default figure placement to htbp
\makeatletter
\def\fps@figure{htbp}
\makeatother
\setlength{\emergencystretch}{3em} % prevent overfull lines
\providecommand{\tightlist}{%
  \setlength{\itemsep}{0pt}\setlength{\parskip}{0pt}}
\setcounter{secnumdepth}{-\maxdimen} % remove section numbering
\ifluatex
  \usepackage{selnolig}  % disable illegal ligatures
\fi

\title{R Notebook}
\author{Kelly Morrow
(\href{mailto:kmorrow@umd.edu}{\nolinkurl{kmorrow@umd.edu}})}
\date{}

\begin{document}
\maketitle

\hypertarget{intro-to-data-wrangling-and-plotting-in-r-tidyverse}{%
\subsubsection{Intro to data-wrangling and plotting in R
tidyverse}\label{intro-to-data-wrangling-and-plotting-in-r-tidyverse}}

\hypertarget{goals-of-this-tutorial}{%
\paragraph{Goals of this tutorial:}\label{goals-of-this-tutorial}}

\begin{itemize}
\tightlist
\item
  Get a basic grasp of popular R libraries to wrangle data
\item
  Quickly create nice plots from said data
\item
  Give examples to apply to your own data
\end{itemize}

\hypertarget{resources}{%
\paragraph{Resources}\label{resources}}

\begin{itemize}
\item
  Accompanying PowerPoint presentation
\item
  List of links to cool pages for inspiration and learning
\item
  Dataset used in this tutorial so that you're able to run the code
  yourself
\end{itemize}

The first thing we'll do is import libraries we need for analysis. Basic
R is OK for plotting but is slow and potentially really frustrating when
data wrangling\ldots{} And that's frustrating to begin with!

\hypertarget{tidyverse}{%
\paragraph{Tidyverse}\label{tidyverse}}

Tidyverse is a suit of packages created for data science. You could
import the packages included separately (will take less time to load),
but I am generally just lazy enough to import tidyverse.~

More
information:~\href{https://www.tidyverse.org/}{\textless https://www.tidyverse.org\textgreater{}}

We'll be utilizing:

\begin{enumerate}
\def\labelenumi{\arabic{enumi}.}
\item
  \href{https://dplyr.tidyverse.org/}{dplyr}
\item
  \href{https://ggplot2.tidyverse.org/}{ggplot2}
\end{enumerate}

\begin{Shaded}
\begin{Highlighting}[]
\CommentTok{\# if running on your own machine that does not have tidyverse installed, uncomment and run the following line. Otherwise no need.}

\CommentTok{\#install.packages(\textquotesingle{}tidyverse\textquotesingle{})}
\CommentTok{\#install.package(\textquotesingle{}viridis\textquotesingle{})         \# cool color schemes that are color{-}blind compatible}

\FunctionTok{library}\NormalTok{(tidyverse)}
\end{Highlighting}
\end{Shaded}

\begin{verbatim}
## -- Attaching packages --------------------------------------- tidyverse 1.3.0 --
\end{verbatim}

\begin{verbatim}
## v ggplot2 3.3.3     v purrr   0.3.4
## v tibble  3.1.0     v dplyr   1.0.4
## v tidyr   1.1.2     v stringr 1.4.0
## v readr   1.4.0     v forcats 0.5.1
\end{verbatim}

\begin{verbatim}
## -- Conflicts ------------------------------------------ tidyverse_conflicts() --
## x dplyr::filter() masks stats::filter()
## x dplyr::lag()    masks stats::lag()
\end{verbatim}

\begin{Shaded}
\begin{Highlighting}[]
\FunctionTok{library}\NormalTok{(viridis)}
\end{Highlighting}
\end{Shaded}

\begin{verbatim}
## Loading required package: viridisLite
\end{verbatim}

\hypertarget{importing-data}{%
\paragraph{Importing data}\label{importing-data}}

R will import nearly any type of data file, but it's important to know
the faults and nuances of your dataframe before attempting (e.g., Do you
have missing data? How are they marked? Do you have numerical values?
Characters? Both?)

Today we'll import a .CSV file using
\texttt{read.delim()}\textsuperscript{1} to a variable called
\textbf{df.}

\textsubscript{1~You~can~use~\texttt{read.table()}~to~import~other~datasets,~but~read.delim()~makes~assumptions~about~the~file~(e.g.,~it's~common~delimited\ldots)~that~makes~it~a~quicker~import.~Text~files~should~be~imported~with~read.table().}

We see that there are some missing values in this table (Charmander is
only Fire type, duh\ldots), so we should replace these missing values
with N/A so that they're easy to ignore and don't get lost in the
shuffle.

We can raise the na.strings argument.

\begin{Shaded}
\begin{Highlighting}[]
\NormalTok{df }\OtherTok{\textless{}{-}} \FunctionTok{read.delim}\NormalTok{(}\StringTok{"Pokemon.csv"}\NormalTok{,                  }\CommentTok{\# if data is located in another directory, you need to specify the path here. }
                   \AttributeTok{sep =} \StringTok{","}\NormalTok{,                    }\CommentTok{\# tell R cells are separated by commas}
                   \AttributeTok{na.strings =} \FunctionTok{c}\NormalTok{(}\StringTok{""}\NormalTok{, }\StringTok{"NA"}\NormalTok{),     }\CommentTok{\# fill missing values with "NA" character}
                   \AttributeTok{stringsAsFactors =}\NormalTok{ F)         }\CommentTok{\# do not automatically make variables factors}


\FunctionTok{head}\NormalTok{(df)                                         }\CommentTok{\# look at first 6 rows of the dataframe}
\end{Highlighting}
\end{Shaded}

\begin{verbatim}
##   X.                  Name Type.1 Type.2 HP Attack Defense Sp..Atk Sp..Def
## 1  1             Bulbasaur  Grass Poison 45     49      49      65      65
## 2  2               Ivysaur  Grass Poison 60     62      63      80      80
## 3  3              Venusaur  Grass Poison 80     82      83     100     100
## 4  3 VenusaurMega Venusaur  Grass Poison 80    100     123     122     120
## 5  4            Charmander   Fire   <NA> 39     52      43      60      50
## 6  5            Charmeleon   Fire   <NA> 58     64      58      80      65
##   Speed Generation Legendary
## 1    45          1     FALSE
## 2    60          1     FALSE
## 3    80          1     FALSE
## 4    80          1     FALSE
## 5    65          1     FALSE
## 6    80          1     FALSE
\end{verbatim}

Better! Now we can easily ignore these cells.

\hypertarget{dplyr}{%
\subsubsection{Dplyr}\label{dplyr}}

\texttt{dplyr} uses a variety of verbs and pipes
(\texttt{\%\textgreater{}\%}) to transform your data.

\textbf{What is a pipe?}

\begin{itemize}
\item
  Think of a pipe as saying ``and then'' between command lines.
\item
  This allows you to string multiple verbs together to get your ideal
  end product
\item
  Quicker than utilizing \texttt{for} and \texttt{if/else} loops
\end{itemize}

\hypertarget{dplyr-verbs-to-remember}{%
\paragraph{Dplyr verbs to remember}\label{dplyr-verbs-to-remember}}

\begin{longtable}[]{@{}ll@{}}
\toprule
term & description \\
\midrule
\endhead
\texttt{select()} & select specific columns in dataframe \\
\texttt{filter()} & filter rows of specific values \\
\texttt{arrange()} & re-order dataframe based on a row \\
\texttt{mutate()} & create a new column \\
\texttt{summarise()} & summarize column values \\
\texttt{group\_by()} & allow for group operations \\
\bottomrule
\end{longtable}

\hypertarget{using-dplyr-to-get-summary-statistics}{%
\subsubsection{Using dplyr to get summary
statistics}\label{using-dplyr-to-get-summary-statistics}}

dplyr follows a particular formatting where the dataframe is first
specified, followed by a pipe, followed by whatever transformations you
would like to make.

\texttt{df\ \%\textgreater{}\%\ transformation()}

To save our changes to a variable, we just assign it to value as we did
initially

\texttt{df.cool\ \textless{}-\ df\ \%\textgreater{}\%\ transformation()}

\hypertarget{lets-say-we-were-interested-in-summarizing-how-attack-stats-differ-between-pokemon-types-but-only-for-generation-1.}{%
\paragraph{Let's say we were interested in summarizing how Attack stats
differ between Pokemon types, but only for Generation
1.}\label{lets-say-we-were-interested-in-summarizing-how-attack-stats-differ-between-pokemon-types-but-only-for-generation-1.}}

Steps:

\begin{enumerate}
\def\labelenumi{\arabic{enumi}.}
\tightlist
\item
  Select columns wanted to perform actions (optional, just makes things
  neater)
\item
  Filter rows where Generation value equals 1
\item
  Group by the Type.1 variable
\item
  Calculate the number of Pokemon within each Type, their mean Attack,
  and standard deviation Attack
\end{enumerate}

\begin{Shaded}
\begin{Highlighting}[]
\NormalTok{summary }\OtherTok{\textless{}{-}}\NormalTok{  df }\SpecialCharTok{\%\textgreater{}\%}                          \CommentTok{\# assign our Attack stats to a new dataframe}
  
            \FunctionTok{select}\NormalTok{(}\SpecialCharTok{{-}}\NormalTok{X.) }\SpecialCharTok{\%\textgreater{}\%}                 \CommentTok{\# select all columns EXCEPT X.}
  
            \FunctionTok{filter}\NormalTok{(Generation }\SpecialCharTok{==} \DecValTok{1}\NormalTok{) }\SpecialCharTok{\%\textgreater{}\%}     \CommentTok{\# only select Pokemon in Gen. 1}
  
            \FunctionTok{group\_by}\NormalTok{(Type}\FloatTok{.1}\NormalTok{) }\SpecialCharTok{\%\textgreater{}\%}            \CommentTok{\# Group Pokemon by their Type to create averages}
  
            \FunctionTok{summarize}\NormalTok{(}\AttributeTok{n =} \FunctionTok{n}\NormalTok{(),              }\CommentTok{\# Create a summary table that includes a count}
              \AttributeTok{mean.Attack =} \FunctionTok{mean}\NormalTok{(Attack),          }\CommentTok{\# mean}
              \AttributeTok{sd.Attack =} \FunctionTok{sd}\NormalTok{(Attack))              }\CommentTok{\# and standard deviation}

\NormalTok{summary}
\end{Highlighting}
\end{Shaded}

\begin{verbatim}
## # A tibble: 15 x 4
##    Type.1       n mean.Attack sd.Attack
##  * <chr>    <int>       <dbl>     <dbl>
##  1 Bug         14        76.4      45.5
##  2 Dragon       3        94        36.1
##  3 Electric     9        62        22.3
##  4 Fairy        2        57.5      17.7
##  5 Fighting     7       103.       18.7
##  6 Fire        14        88.6      26.6
##  7 Ghost        4        53.8      14.4
##  8 Grass       13        72.9      21.1
##  9 Ground       8        81.9      25.2
## 10 Ice          2        67.5      24.7
## 11 Normal      24        70.6      26.6
## 12 Poison      14        74.4      19.2
## 13 Psychic     11        79.2      52.9
## 14 Rock        10        87.5      32.3
## 15 Water       31        74.2      29.4
\end{verbatim}

\hypertarget{creating-new-variables}{%
\subsubsection{Creating new variables}\label{creating-new-variables}}

Often we want to create new columns to our existing dataset. For
instance, we might want to know the total value of each Pokemon. This
can easily be calculated using \texttt{mutate()}.

\begin{Shaded}
\begin{Highlighting}[]
\NormalTok{df }\OtherTok{\textless{}{-}}\NormalTok{ df }\SpecialCharTok{\%\textgreater{}\%}                               \CommentTok{\# just updating the current dataframe}
      \FunctionTok{group\_by}\NormalTok{(Name) }\SpecialCharTok{\%\textgreater{}\%}                   \CommentTok{\# We want to calculate the total for each Pokemon}
      \FunctionTok{mutate}\NormalTok{(}\AttributeTok{Total =} \FunctionTok{sum}\NormalTok{(HP, Attack, Defense, Sp..Atk, Sp..Def, Speed)) }\CommentTok{\# new variable Total}

\CommentTok{\# side note: if you need to ever implement just one dplyr verb, you can do so without pipes!}

\FunctionTok{mutate}\NormalTok{(df, }\AttributeTok{Total =} \FunctionTok{sum}\NormalTok{(HP, Attack, Defense, Sp..Atk, Sp..Def, Speed))}
\end{Highlighting}
\end{Shaded}

\begin{verbatim}
## # A tibble: 800 x 13
## # Groups:   Name [800]
##       X. Name           Type.1 Type.2    HP Attack Defense Sp..Atk Sp..Def Speed
##    <int> <chr>          <chr>  <chr>  <int>  <int>   <int>   <int>   <int> <int>
##  1     1 Bulbasaur      Grass  Poison    45     49      49      65      65    45
##  2     2 Ivysaur        Grass  Poison    60     62      63      80      80    60
##  3     3 Venusaur       Grass  Poison    80     82      83     100     100    80
##  4     3 VenusaurMega ~ Grass  Poison    80    100     123     122     120    80
##  5     4 Charmander     Fire   <NA>      39     52      43      60      50    65
##  6     5 Charmeleon     Fire   <NA>      58     64      58      80      65    80
##  7     6 Charizard      Fire   Flying    78     84      78     109      85   100
##  8     6 CharizardMega~ Fire   Dragon    78    130     111     130      85   100
##  9     6 CharizardMega~ Fire   Flying    78    104      78     159     115   100
## 10     7 Squirtle       Water  <NA>      44     48      65      50      64    43
## # ... with 790 more rows, and 3 more variables: Generation <int>,
## #   Legendary <lgl>, Total <int>
\end{verbatim}

We also can use conditional statements to create new variables based on
existing columns. Perhaps we want to categorized Pokemon based on
whether they have greater Defense or Attack stats.

With real datasets we would probably have finer classifications, but
this is just an example.

Because we have two options (Defensive/Offensive), we can use
\texttt{mutate()} and \texttt{ifelse()}

We can then choose to only look at Pokemon who are considered Defensive
using \texttt{filter()}

\begin{Shaded}
\begin{Highlighting}[]
\NormalTok{df }\SpecialCharTok{\%\textgreater{}\%} 
      \FunctionTok{group\_by}\NormalTok{(Name) }\SpecialCharTok{\%\textgreater{}\%}
      \FunctionTok{mutate}\NormalTok{(}\AttributeTok{DO =} \FunctionTok{ifelse}\NormalTok{(Defense }\SpecialCharTok{\textgreater{}}\NormalTok{ Attack,}\StringTok{"Defensive"}\NormalTok{, }\StringTok{"Offensive"}\NormalTok{)) }\SpecialCharTok{\%\textgreater{}\%}
      \FunctionTok{filter}\NormalTok{(DO }\SpecialCharTok{==} \StringTok{"Defensive"}\NormalTok{)}
\end{Highlighting}
\end{Shaded}

\begin{verbatim}
## # A tibble: 298 x 14
## # Groups:   Name [298]
##       X. Name           Type.1 Type.2    HP Attack Defense Sp..Atk Sp..Def Speed
##    <int> <chr>          <chr>  <chr>  <int>  <int>   <int>   <int>   <int> <int>
##  1     2 Ivysaur        Grass  Poison    60     62      63      80      80    60
##  2     3 Venusaur       Grass  Poison    80     82      83     100     100    80
##  3     3 VenusaurMega ~ Grass  Poison    80    100     123     122     120    80
##  4     7 Squirtle       Water  <NA>      44     48      65      50      64    43
##  5     8 Wartortle      Water  <NA>      59     63      80      65      80    58
##  6     9 Blastoise      Water  <NA>      79     83     100      85     105    78
##  7     9 BlastoiseMega~ Water  <NA>      79    103     120     135     115    78
##  8    10 Caterpie       Bug    <NA>      45     30      35      20      20    45
##  9    11 Metapod        Bug    <NA>      50     20      55      25      25    30
## 10    12 Butterfree     Bug    Flying    60     45      50      90      80    70
## # ... with 288 more rows, and 4 more variables: Generation <int>,
## #   Legendary <lgl>, Total <int>, DO <chr>
\end{verbatim}

As mentioned before, we can either save the dataframe with these new
variables, but if we're just grouping to plot, we might not want to.
Otherwise we might end up with df1, df2, df3\ldots{} Instead, we can
move straight from using dplyr to plotting with ggplot using a pipe.

\begin{Shaded}
\begin{Highlighting}[]
\NormalTok{df }\SpecialCharTok{\%\textgreater{}\%} 
  \FunctionTok{mutate}\NormalTok{(}\AttributeTok{DO =} \FunctionTok{ifelse}\NormalTok{(Defense }\SpecialCharTok{\textgreater{}}\NormalTok{ Attack,}\StringTok{"Defensive"}\NormalTok{, }\StringTok{"Offensive"}\NormalTok{)) }\SpecialCharTok{\%\textgreater{}\%}
  \FunctionTok{filter}\NormalTok{(DO }\SpecialCharTok{==} \StringTok{"Defensive"}\NormalTok{) }\SpecialCharTok{\%\textgreater{}\%}
  \FunctionTok{ggplot}\NormalTok{(}\FunctionTok{aes}\NormalTok{(}\AttributeTok{x =}\NormalTok{ Defense, }\AttributeTok{y =}\NormalTok{ Attack)) }\SpecialCharTok{+}
  \FunctionTok{geom\_point}\NormalTok{()}
\end{Highlighting}
\end{Shaded}

\includegraphics{tidyverse-tutorial_files/figure-latex/unnamed-chunk-6-1.pdf}

\hypertarget{ggplot-basics}{%
\subsubsection{ggplot basics}\label{ggplot-basics}}

\texttt{ggplot()} creates plots in layers

\begin{itemize}
\item
  visual items are referred to as \textbf{geoms}, short for geometric
  objects

  \begin{itemize}
  \item
    geoms answer the question of \emph{how} we want to plot our data
  \item
    geoms come in many shapes and we can add more than one to our plot
  \end{itemize}
\item
  aesthetics of the plot can be controlled with\ldots{} aesthetics,
  which are referred to as \textbf{aes}

  \begin{itemize}
  \item
    \texttt{aes()} stands for \textbf{aesthetic}, or ``something you can
    see.'' Each aesthetic used maps a visual cue to a variable
    (\url{https://beanumber.github.io/sds192/lab-ggplot2.html})
  \item
    Can be used to vary colors, shapes, sizes, transparencies\ldots etc

    \begin{itemize}
    \tightlist
    \item
      Some geoms have specific aesthetics that work only for them. You
      can always check by typing
      \texttt{?geom\_\textless{}NAME\textgreater{}()} into your command
      line, replacing \textless NAME\textgreater{} with your geom of
      choice.
    \end{itemize}
  \end{itemize}
\end{itemize}

ggplot follows the following formula where \textless\textgreater{}
indicates input from the user:

\begin{verbatim}
`ggplot(<dataset>, aes(x = <x.variable>, y = <y.variable>)`
\end{verbatim}

After specifying our dataset, X and Y variables we can tell ggplot
\emph{how} we want to display the data using a \texttt{geom} command.
Notice that ggplot does not use pipes, but \texttt{+} as an indicator
that the next line is included in the plot.

\begin{verbatim}
    ggplot(<dataset>, aes(x = <x.variable>, y = <y.variable>) +
      geom_point()
\end{verbatim}

Basic examples:

\begin{Shaded}
\begin{Highlighting}[]
\FunctionTok{ggplot}\NormalTok{(df, }\FunctionTok{aes}\NormalTok{(}\AttributeTok{x =}\NormalTok{ Attack, }\AttributeTok{y =}\NormalTok{ Defense)) }\SpecialCharTok{+}
  \FunctionTok{geom\_point}\NormalTok{()}
\end{Highlighting}
\end{Shaded}

\includegraphics{tidyverse-tutorial_files/figure-latex/unnamed-chunk-7-1.pdf}

\begin{Shaded}
\begin{Highlighting}[]
\FunctionTok{ggplot}\NormalTok{(df, }\FunctionTok{aes}\NormalTok{(}\AttributeTok{x =}\NormalTok{ Type}\FloatTok{.1}\NormalTok{, }\AttributeTok{y =}\NormalTok{ Attack)) }\SpecialCharTok{+}
  \FunctionTok{geom\_boxplot}\NormalTok{()}
\end{Highlighting}
\end{Shaded}

\includegraphics{tidyverse-tutorial_files/figure-latex/unnamed-chunk-8-1.pdf}

Even without aesthetics we can make pretty decent basic plots. But
aesthetics will help us draw attention to what matters in our
visualizations.

\hypertarget{heres-where-it-gets-a-teeny-bit-tricky}{%
\paragraph{Here's where it gets a teeny bit
tricky\ldots{}}\label{heres-where-it-gets-a-teeny-bit-tricky}}

Aesthetics can belong either inside or outside parentheses and this
changes what they impact.

\begin{itemize}
\item
  Aesthetics placed \textbf{inside} parentheses set the aesthetic based
  on a variable.

  \begin{itemize}
  \tightlist
  \item
    For example, if we wanted colors to be different for Pokemon types.
  \end{itemize}
\end{itemize}

\begin{Shaded}
\begin{Highlighting}[]
\CommentTok{\# impact of specifying inside parentheses}

\FunctionTok{ggplot}\NormalTok{(df, }\FunctionTok{aes}\NormalTok{(}\AttributeTok{x =}\NormalTok{ Attack, }\AttributeTok{y =}\NormalTok{ Defense)) }\SpecialCharTok{+}
  \FunctionTok{geom\_point}\NormalTok{(}\FunctionTok{aes}\NormalTok{(}\AttributeTok{color =}\NormalTok{ Type}\FloatTok{.1}\NormalTok{))}
\end{Highlighting}
\end{Shaded}

\includegraphics{tidyverse-tutorial_files/figure-latex/unnamed-chunk-9-1.pdf}

\begin{itemize}
\item
  Aesthetics placed \textbf{outside} of parentheses set the aesthetic to
  a specific value

  \begin{itemize}
  \tightlist
  \item
    If we wanted all points to be red instead of the default black.
  \end{itemize}
\end{itemize}

\begin{Shaded}
\begin{Highlighting}[]
\CommentTok{\# impact of setting outside of parentheses}
\FunctionTok{ggplot}\NormalTok{(df, }\FunctionTok{aes}\NormalTok{(}\AttributeTok{x =}\NormalTok{ Attack, }\AttributeTok{y =}\NormalTok{ Defense)) }\SpecialCharTok{+}
  \FunctionTok{geom\_point}\NormalTok{(}\AttributeTok{color =} \StringTok{"red"}\NormalTok{)}
\end{Highlighting}
\end{Shaded}

\includegraphics{tidyverse-tutorial_files/figure-latex/unnamed-chunk-10-1.pdf}

Aesthetics can also be specific to a geom (as shown above), or set
globally to impact all geoms.

\begin{Shaded}
\begin{Highlighting}[]
\CommentTok{\# setting a global aesthetic}

\FunctionTok{ggplot}\NormalTok{(df, }\FunctionTok{aes}\NormalTok{(}\AttributeTok{x =}\NormalTok{ Type}\FloatTok{.1}\NormalTok{, }\AttributeTok{y =}\NormalTok{ Attack, }\AttributeTok{color =}\NormalTok{ Type}\FloatTok{.1}\NormalTok{)) }\SpecialCharTok{+}
  \FunctionTok{geom\_boxplot}\NormalTok{() }\SpecialCharTok{+} 
  \FunctionTok{geom\_jitter}\NormalTok{()}
\end{Highlighting}
\end{Shaded}

\includegraphics{tidyverse-tutorial_files/figure-latex/unnamed-chunk-11-1.pdf}

Yikes, that is maybe a bit of an eyesore, but you get the idea.
Sometimes it's better to set aesthetics to a specific geom instead!

\begin{Shaded}
\begin{Highlighting}[]
\FunctionTok{ggplot}\NormalTok{(df, }\FunctionTok{aes}\NormalTok{(}\AttributeTok{x =}\NormalTok{ Type}\FloatTok{.1}\NormalTok{, }\AttributeTok{y =}\NormalTok{ Attack)) }\SpecialCharTok{+}
  \FunctionTok{geom\_boxplot}\NormalTok{() }\SpecialCharTok{+} 
  \FunctionTok{geom\_jitter}\NormalTok{(}\FunctionTok{aes}\NormalTok{(}\AttributeTok{color =}\NormalTok{ Type}\FloatTok{.1}\NormalTok{))}
\end{Highlighting}
\end{Shaded}

\includegraphics{tidyverse-tutorial_files/figure-latex/unnamed-chunk-12-1.pdf}

We can take this even further by modifying the plot with \emph{more}
aesthetics outside of parentheses.

We can manipulate:

\begin{itemize}
\item
  transparency using \texttt{alpha}
\item
  size using \texttt{size}
\item
  type of line used \texttt{linetype}
\item
  shape of a point \texttt{point}
\end{itemize}

And the list continues! Again, you can always check what aesthetics are
available to specific geom by checking its help page.

For example, we can use the aesthetic \texttt{width} for
\texttt{geom\_jitter()} (scatterplot) to set the width of the jitter,
but we would get an error if we tried to use it for
\texttt{geom\_point()}.

\begin{Shaded}
\begin{Highlighting}[]
\FunctionTok{ggplot}\NormalTok{(df, }\FunctionTok{aes}\NormalTok{(}\AttributeTok{x =}\NormalTok{ Type}\FloatTok{.1}\NormalTok{, }\AttributeTok{y =}\NormalTok{ Attack)) }\SpecialCharTok{+}
  \FunctionTok{geom\_boxplot}\NormalTok{(}\AttributeTok{alpha =} \FloatTok{0.5}\NormalTok{,                     }\CommentTok{\# make boxplots transparent}
               \AttributeTok{size =} \FloatTok{0.75}\NormalTok{) }\SpecialCharTok{+}                   \CommentTok{\# make outlines a little thicker}
  \FunctionTok{geom\_jitter}\NormalTok{(}\FunctionTok{aes}\NormalTok{(}\AttributeTok{color =}\NormalTok{ Type}\FloatTok{.1}\NormalTok{),}
              \AttributeTok{width =} \FloatTok{0.25}\NormalTok{,                     }\CommentTok{\# set the width of the jitter}
              \AttributeTok{size =} \FloatTok{0.5}\NormalTok{)                       }\CommentTok{\# make points smaller than default}
\end{Highlighting}
\end{Shaded}

\includegraphics{tidyverse-tutorial_files/figure-latex/unnamed-chunk-13-1.pdf}

\hypertarget{labels}{%
\paragraph{Labels}\label{labels}}

Labels can be specified pretty easily with the \texttt{lab()} option,
allowing us to override the default which are just the column names from
the dataframe.

\begin{Shaded}
\begin{Highlighting}[]
\FunctionTok{ggplot}\NormalTok{(df, }\FunctionTok{aes}\NormalTok{(}\AttributeTok{x =}\NormalTok{ Type}\FloatTok{.1}\NormalTok{, }\AttributeTok{y =}\NormalTok{ Attack)) }\SpecialCharTok{+}
  \FunctionTok{geom\_boxplot}\NormalTok{(}\AttributeTok{alpha =} \FloatTok{0.5}\NormalTok{,                     }
               \AttributeTok{size =} \FloatTok{0.75}\NormalTok{) }\SpecialCharTok{+}                   
  \FunctionTok{geom\_jitter}\NormalTok{(}\FunctionTok{aes}\NormalTok{(}\AttributeTok{color =}\NormalTok{ Type}\FloatTok{.1}\NormalTok{),}
              \AttributeTok{width =} \FloatTok{0.25}\NormalTok{,                     }
              \AttributeTok{size =} \FloatTok{0.5}\NormalTok{) }\SpecialCharTok{+}
  \FunctionTok{labs}\NormalTok{(}\AttributeTok{x =} \StringTok{"Pokemon type"}\NormalTok{,                       }\CommentTok{\# x{-}axis label}
       \AttributeTok{y =} \StringTok{"Attack statistic"}\NormalTok{,                   }\CommentTok{\# y{-}axis label}
       \AttributeTok{title =} \StringTok{"Attack stats by Pokemon type"}\NormalTok{)   }\CommentTok{\# plot title}
\end{Highlighting}
\end{Shaded}

\includegraphics{tidyverse-tutorial_files/figure-latex/unnamed-chunk-14-1.pdf}

\hypertarget{editing-the-global-theme}{%
\paragraph{Editing the global theme}\label{editing-the-global-theme}}

There are obviously some unsightly things we should take care of

\begin{itemize}
\item
  overlap of x-axis text :(
\item
  redundant legend :(
\item
  ugly gray default background :(
\end{itemize}

We can make general changes to the plot by altering the plot's theme.
Here you can change almost anything you can imagine! If you don't
believe me, type \texttt{?theme()} and take a look at the list.

\begin{Shaded}
\begin{Highlighting}[]
\FunctionTok{ggplot}\NormalTok{(df, }\FunctionTok{aes}\NormalTok{(}\AttributeTok{x =}\NormalTok{ Type}\FloatTok{.1}\NormalTok{, }\AttributeTok{y =}\NormalTok{ Attack)) }\SpecialCharTok{+}
  \FunctionTok{geom\_boxplot}\NormalTok{(}\AttributeTok{alpha =} \FloatTok{0.5}\NormalTok{,                     }
               \AttributeTok{size =} \FloatTok{0.75}\NormalTok{) }\SpecialCharTok{+}                   
  \FunctionTok{geom\_jitter}\NormalTok{(}\FunctionTok{aes}\NormalTok{(}\AttributeTok{color =}\NormalTok{ Type}\FloatTok{.1}\NormalTok{),}
              \AttributeTok{width =} \FloatTok{0.25}\NormalTok{,                     }
              \AttributeTok{size =} \FloatTok{0.5}\NormalTok{) }\SpecialCharTok{+}
  \FunctionTok{labs}\NormalTok{(}\AttributeTok{x =} \StringTok{"Pokemon type"}\NormalTok{,                       }
       \AttributeTok{y =} \StringTok{"Attack statistic"}\NormalTok{,                   }
       \AttributeTok{title =} \StringTok{"Attack stats by Pokemon type"}\NormalTok{) }\SpecialCharTok{+}
  \FunctionTok{theme}\NormalTok{(}\AttributeTok{panel.background =} \FunctionTok{element\_rect}\NormalTok{(}\AttributeTok{fill =} \StringTok{"white"}\NormalTok{),  }\CommentTok{\# change from unsightly gray to white}
        \AttributeTok{legend.position =} \StringTok{\textquotesingle{}none\textquotesingle{}}\NormalTok{,                         }\CommentTok{\# begone useless legend!}
        \AttributeTok{axis.text.x =} \FunctionTok{element\_text}\NormalTok{(}\AttributeTok{angle =} \DecValTok{45}\NormalTok{)            }\CommentTok{\# put x text on an angle}
\NormalTok{  )}
\end{Highlighting}
\end{Shaded}

\includegraphics{tidyverse-tutorial_files/figure-latex/unnamed-chunk-15-1.pdf}

\hypertarget{putting-dplyr-and-ggplot-together}{%
\subsubsection{Putting dplyr and ggplot
together}\label{putting-dplyr-and-ggplot-together}}

Sometimes we need to make specific transformations to our dataframe
before we're able to plot. One such plot is the stacked bar plot. So
let's pretend we wanted to visualize how total stat points are broken
down by Pokemon type.

Steps:

\begin{enumerate}
\def\labelenumi{\arabic{enumi}.}
\tightlist
\item
  Create a dataframe that includes averages of each stat for each
  Pokemon type
\item
  Transpose the data from wide to long format
\item
  Plot!
\end{enumerate}

\begin{Shaded}
\begin{Highlighting}[]
\CommentTok{\# 1. create a dataframe of averages}

\NormalTok{df }\SpecialCharTok{\%\textgreater{}\%} 
  \FunctionTok{group\_by}\NormalTok{(Type}\FloatTok{.1}\NormalTok{) }\SpecialCharTok{\%\textgreater{}\%}
  \FunctionTok{summarize}\NormalTok{(}\AttributeTok{Avg.Total =} \FunctionTok{mean}\NormalTok{(Total),}
            \AttributeTok{Avg.HP =} \FunctionTok{mean}\NormalTok{(HP),}
            \AttributeTok{Avg.Attack =} \FunctionTok{mean}\NormalTok{(Attack),}
            \AttributeTok{Avg.Defense =} \FunctionTok{mean}\NormalTok{(Defense),}
            \AttributeTok{Avg.SpAtk =} \FunctionTok{mean}\NormalTok{(Sp..Atk),}
            \AttributeTok{Avg.SpDef =} \FunctionTok{mean}\NormalTok{(Sp..Def)) }
\end{Highlighting}
\end{Shaded}

\begin{verbatim}
## # A tibble: 18 x 7
##    Type.1   Avg.Total Avg.HP Avg.Attack Avg.Defense Avg.SpAtk Avg.SpDef
##  * <chr>        <dbl>  <dbl>      <dbl>       <dbl>     <dbl>     <dbl>
##  1 Bug           379.   56.9       71.0        70.7      53.9      64.8
##  2 Dark          446.   66.8       88.4        70.2      74.6      69.5
##  3 Dragon        551.   83.3      112.         86.4      96.8      88.8
##  4 Electric      443.   59.8       69.1        66.3      90.0      73.7
##  5 Fairy         413.   74.1       61.5        65.7      78.5      84.7
##  6 Fighting      416.   69.9       96.8        65.9      53.1      64.7
##  7 Fire          458.   69.9       84.8        67.8      89.0      72.2
##  8 Flying        485    70.8       78.8        66.2      94.2      72.5
##  9 Ghost         440.   64.4       73.8        81.2      79.3      76.5
## 10 Grass         421.   67.3       73.2        70.8      77.5      70.4
## 11 Ground        438.   73.8       95.8        84.8      56.5      62.8
## 12 Ice           433.   72         72.8        71.4      77.5      76.3
## 13 Normal        402.   77.3       73.5        59.8      55.8      63.7
## 14 Poison        399.   67.2       74.7        68.8      60.4      64.4
## 15 Psychic       476.   70.6       71.5        67.7      98.4      86.3
## 16 Rock          454.   65.4       92.9       101.       63.3      75.5
## 17 Steel         488.   65.2       92.7       126.       67.5      80.6
## 18 Water         430.   72.1       74.2        72.9      74.8      70.5
\end{verbatim}

Transposing the dataframe uses a new verb we haven't gone over,
\texttt{gather()} which takes three values:

\begin{itemize}
\item
  \textbf{Key:} the name of your new column that will store the old
  column names
\item
  \textbf{Value}: name of new column that will hold all of the old
  column values
\item
  Which columns you'd like to include in the transpose (otherwise it
  will transpose all!)
\end{itemize}

\begin{Shaded}
\begin{Highlighting}[]
\CommentTok{\# 2. Transpose data from wide to long form}

\NormalTok{df }\SpecialCharTok{\%\textgreater{}\%} 
  \FunctionTok{group\_by}\NormalTok{(Type}\FloatTok{.1}\NormalTok{) }\SpecialCharTok{\%\textgreater{}\%}
  \FunctionTok{summarize}\NormalTok{(}\AttributeTok{Avg.Total =} \FunctionTok{mean}\NormalTok{(Total),}
            \AttributeTok{Avg.HP =} \FunctionTok{mean}\NormalTok{(HP),}
            \AttributeTok{Avg.Attack =} \FunctionTok{mean}\NormalTok{(Attack),}
            \AttributeTok{Avg.Defense =} \FunctionTok{mean}\NormalTok{(Defense),}
            \AttributeTok{Avg.SpAtk =} \FunctionTok{mean}\NormalTok{(Sp..Atk),}
            \AttributeTok{Avg.SpDef =} \FunctionTok{mean}\NormalTok{(Sp..Def)) }\SpecialCharTok{\%\textgreater{}\%}
  \FunctionTok{gather}\NormalTok{(}\AttributeTok{key =} \StringTok{"Stat"}\NormalTok{, }\AttributeTok{value =} \StringTok{"Average"}\NormalTok{, Avg.HP}\SpecialCharTok{:}\NormalTok{Avg.SpDef)}
\end{Highlighting}
\end{Shaded}

\begin{verbatim}
## # A tibble: 90 x 4
##    Type.1   Avg.Total Stat   Average
##    <chr>        <dbl> <chr>    <dbl>
##  1 Bug           379. Avg.HP    56.9
##  2 Dark          446. Avg.HP    66.8
##  3 Dragon        551. Avg.HP    83.3
##  4 Electric      443. Avg.HP    59.8
##  5 Fairy         413. Avg.HP    74.1
##  6 Fighting      416. Avg.HP    69.9
##  7 Fire          458. Avg.HP    69.9
##  8 Flying        485  Avg.HP    70.8
##  9 Ghost         440. Avg.HP    64.4
## 10 Grass         421. Avg.HP    67.3
## # ... with 80 more rows
\end{verbatim}

\begin{Shaded}
\begin{Highlighting}[]
\CommentTok{\# 3. Make a fancy plot! (this is where we\textquotesingle{}ll use the viridis package)}

\NormalTok{df }\SpecialCharTok{\%\textgreater{}\%} 
  \FunctionTok{group\_by}\NormalTok{(Type}\FloatTok{.1}\NormalTok{) }\SpecialCharTok{\%\textgreater{}\%}
  \FunctionTok{summarize}\NormalTok{(}\AttributeTok{Avg.Total =} \FunctionTok{mean}\NormalTok{(Total),}
            \AttributeTok{Avg.HP =} \FunctionTok{mean}\NormalTok{(HP),}
            \AttributeTok{Avg.Attack =} \FunctionTok{mean}\NormalTok{(Attack),}
            \AttributeTok{Avg.Defense =} \FunctionTok{mean}\NormalTok{(Defense),}
            \AttributeTok{Avg.SpAtk =} \FunctionTok{mean}\NormalTok{(Sp..Atk),}
            \AttributeTok{Avg.SpDef =} \FunctionTok{mean}\NormalTok{(Sp..Def)) }\SpecialCharTok{\%\textgreater{}\%}
  \FunctionTok{gather}\NormalTok{(}\AttributeTok{key =} \StringTok{"Stat"}\NormalTok{, }\AttributeTok{value =} \StringTok{"Average"}\NormalTok{, Avg.HP}\SpecialCharTok{:}\NormalTok{Avg.SpDef) }\SpecialCharTok{\%\textgreater{}\%}  
  
  \CommentTok{\# transfer right into ggplot with a pipe !}
  
  \FunctionTok{ggplot}\NormalTok{(}\FunctionTok{aes}\NormalTok{(}\AttributeTok{x =}\NormalTok{ Type}\FloatTok{.1}\NormalTok{, }\AttributeTok{y =}\NormalTok{ Avg.Total, }\AttributeTok{fill =}\NormalTok{ Stat)) }\SpecialCharTok{+}
  
  \FunctionTok{geom\_bar}\NormalTok{(}\AttributeTok{stat =} \StringTok{\textquotesingle{}identity\textquotesingle{}}\NormalTok{, }\AttributeTok{color =} \StringTok{\textquotesingle{}black\textquotesingle{}}\NormalTok{) }\SpecialCharTok{+} \CommentTok{\# identity just means no transforms}
  
  \FunctionTok{labs}\NormalTok{(}\AttributeTok{x =} \StringTok{"Pokemon Type"}\NormalTok{,}
       \AttributeTok{y =} \StringTok{"Average total stat points"}\NormalTok{,}
       \AttributeTok{title =} \StringTok{"Composition of total stats by Pokemon type"}\NormalTok{,}
       \AttributeTok{fill =} \StringTok{"Stat"}\NormalTok{) }\SpecialCharTok{+}
  
  \FunctionTok{coord\_cartesian}\NormalTok{(}\AttributeTok{ylim =} \FunctionTok{c}\NormalTok{(}\DecValTok{0}\NormalTok{, }\DecValTok{3000}\NormalTok{)) }\SpecialCharTok{+}          \CommentTok{\# customize plot limits }
  
  \FunctionTok{scale\_fill\_viridis}\NormalTok{(}\AttributeTok{discrete =} \ConstantTok{TRUE}\NormalTok{) }\SpecialCharTok{+}         \CommentTok{\# our values are not continous}
  \CommentTok{\# notice we\textquotesingle{}re using scale FILL, because we set the fill aesthetic above}
  
  \FunctionTok{scale\_y\_continuous}\NormalTok{(}\AttributeTok{expand =} \FunctionTok{c}\NormalTok{(}\DecValTok{0}\NormalTok{,}\DecValTok{0}\NormalTok{)) }\SpecialCharTok{+}         \CommentTok{\# get rid of unsightly gap at [0,0]}
  
  \FunctionTok{theme\_bw}\NormalTok{() }\SpecialCharTok{+}                                  \CommentTok{\# pre{-}set theme that is not so bad}
  
  \FunctionTok{theme}\NormalTok{(}\AttributeTok{axis.text.x =} \FunctionTok{element\_text}\NormalTok{(}\AttributeTok{angle =} \DecValTok{45}\NormalTok{,  }\CommentTok{\# angle x{-}axis text}
                                   \AttributeTok{hjust =} \DecValTok{1}\NormalTok{))  }\CommentTok{\# slide text down a bit (was peaking into plot)}
\end{Highlighting}
\end{Shaded}

\includegraphics{tidyverse-tutorial_files/figure-latex/unnamed-chunk-18-1.pdf}

\hypertarget{facet-nating-using-facets}{%
\paragraph{Facet-nating (Using
facets)}\label{facet-nating-using-facets}}

Sometimes it's best to display data in subplots, especially when we have
a lot of different levels to a variable

For instance, there is obviously a relationship between Defense and
Attack stats, but we're also intermixing 18 groups! We've also seen that
simply adding colors doesn't help, just too many groups!

\begin{Shaded}
\begin{Highlighting}[]
\FunctionTok{ggplot}\NormalTok{(df, }\FunctionTok{aes}\NormalTok{(}\AttributeTok{x =}\NormalTok{ Attack, }\AttributeTok{y =}\NormalTok{ Defense)) }\SpecialCharTok{+}
  \FunctionTok{geom\_point}\NormalTok{(}\FunctionTok{aes}\NormalTok{(}\AttributeTok{color =}\NormalTok{ Type}\FloatTok{.1}\NormalTok{))}
\end{Highlighting}
\end{Shaded}

\includegraphics{tidyverse-tutorial_files/figure-latex/unnamed-chunk-19-1.pdf}

\begin{Shaded}
\begin{Highlighting}[]
\FunctionTok{ggplot}\NormalTok{(df, }\FunctionTok{aes}\NormalTok{(}\AttributeTok{x =}\NormalTok{ Attack, }\AttributeTok{y =}\NormalTok{ Defense)) }\SpecialCharTok{+}
  \FunctionTok{facet\_wrap}\NormalTok{(}\SpecialCharTok{\textasciitilde{}}\NormalTok{Type}\FloatTok{.1}\NormalTok{,                   }\CommentTok{\# select variable to facet by}
             \AttributeTok{nrow =} \DecValTok{3}\NormalTok{) }\SpecialCharTok{+}                \CommentTok{\# set number of rows of plots}
  
  \FunctionTok{geom\_smooth}\NormalTok{(}\AttributeTok{method =} \StringTok{\textquotesingle{}lm\textquotesingle{}}\NormalTok{,             }\CommentTok{\# add line of best fit}
              \AttributeTok{se =} \ConstantTok{FALSE}\NormalTok{,                }\CommentTok{\# don\textquotesingle{}t include SE ribbon}
              \AttributeTok{size =} \FloatTok{0.75}\NormalTok{) }\SpecialCharTok{+}             \CommentTok{\# make lines a bit more substantial}
  
  \FunctionTok{geom\_point}\NormalTok{(}\FunctionTok{aes}\NormalTok{(}\AttributeTok{color =}\NormalTok{ Type}\FloatTok{.1}\NormalTok{),  }
             \AttributeTok{size =} \FloatTok{0.65}\NormalTok{,) }\SpecialCharTok{+}
  
  \FunctionTok{theme\_bw}\NormalTok{() }\SpecialCharTok{+} 
  
  \FunctionTok{theme}\NormalTok{(}\AttributeTok{legend.position =} \StringTok{\textquotesingle{}none\textquotesingle{}}\NormalTok{)}
\end{Highlighting}
\end{Shaded}

\begin{verbatim}
## `geom_smooth()` using formula 'y ~ x'
\end{verbatim}

\includegraphics{tidyverse-tutorial_files/figure-latex/unnamed-chunk-20-1.pdf}

\textsubscript{*}*\textsubscript{*}*\textsubscript{*}*\textsubscript{*}*\textsubscript{*}*\textsubscript{FANCIFY}*\textsubscript{*}*\textsubscript{*}*\textsubscript{*}*\textsubscript{*}*

Let's display all data in each facet, but highlight each group!

\begin{Shaded}
\begin{Highlighting}[]
\NormalTok{df2 }\OtherTok{\textless{}{-}} \FunctionTok{select}\NormalTok{(df, }\SpecialCharTok{{-}}\NormalTok{Type}\FloatTok{.1}\NormalTok{)                }\CommentTok{\# grab all data besides our normal grouping variable}

\FunctionTok{ggplot}\NormalTok{(df, }\FunctionTok{aes}\NormalTok{(}\AttributeTok{x =}\NormalTok{ Attack, }\AttributeTok{y =}\NormalTok{ Defense)) }\SpecialCharTok{+}
  \FunctionTok{facet\_wrap}\NormalTok{(}\SpecialCharTok{\textasciitilde{}}\NormalTok{Type}\FloatTok{.1}\NormalTok{,                   }
             \AttributeTok{nrow =} \DecValTok{3}\NormalTok{) }\SpecialCharTok{+}
  
  \FunctionTok{geom\_point}\NormalTok{(}\AttributeTok{data =}\NormalTok{ df2,                  }\CommentTok{\# plot our new df as a big blob of gray}
             \AttributeTok{color =} \StringTok{"gray"}\NormalTok{,}
             \AttributeTok{alpha =} \FloatTok{0.35}\NormalTok{,}
             \AttributeTok{size =} \FloatTok{0.50}\NormalTok{) }\SpecialCharTok{+}
  
  \FunctionTok{geom\_smooth}\NormalTok{(}\AttributeTok{method =} \StringTok{\textquotesingle{}lm\textquotesingle{}}\NormalTok{,             }
              \AttributeTok{se =} \ConstantTok{FALSE}\NormalTok{,               }
              \AttributeTok{size =} \FloatTok{0.75}\NormalTok{) }\SpecialCharTok{+}
  
  \FunctionTok{geom\_point}\NormalTok{(}\FunctionTok{aes}\NormalTok{(}\AttributeTok{color =}\NormalTok{ Type}\FloatTok{.1}\NormalTok{),  }
             \AttributeTok{size =} \FloatTok{0.65}\NormalTok{,) }\SpecialCharTok{+}
  
  \FunctionTok{theme\_bw}\NormalTok{() }\SpecialCharTok{+} 
  
  \FunctionTok{theme}\NormalTok{(}\AttributeTok{legend.position =} \StringTok{\textquotesingle{}none\textquotesingle{}}\NormalTok{,           }
        \AttributeTok{strip.background =} \FunctionTok{element\_blank}\NormalTok{()) }\CommentTok{\# get rid of facet title backgrounds}
\end{Highlighting}
\end{Shaded}

\begin{verbatim}
## `geom_smooth()` using formula 'y ~ x'
\end{verbatim}

\includegraphics{tidyverse-tutorial_files/figure-latex/unnamed-chunk-21-1.pdf}

\hypertarget{saving-your-masterpiece}{%
\paragraph{Saving your masterpiece}\label{saving-your-masterpiece}}

Plots can be saved by adding a \texttt{ggsave()} command after your plot

\texttt{ggsave(\textquotesingle{}name-of-file.png\textquotesingle{},\ height\ =\ XXX,\ width\ =\ XXX,\ dpi\ =\ XXX)}

\begin{itemize}
\item
  You can save as almost any image extension (.png, .jpeg, .tiff, pdf)
\item
  Specify dpi (this is handy for publications!
\item
  Specify height and width
\item
  Specify unit (inches, cm, mm)
\end{itemize}

\begin{Shaded}
\begin{Highlighting}[]
\NormalTok{df2 }\OtherTok{\textless{}{-}} \FunctionTok{select}\NormalTok{(df, }\SpecialCharTok{{-}}\NormalTok{Type}\FloatTok{.1}\NormalTok{)               }

\FunctionTok{ggplot}\NormalTok{(df, }\FunctionTok{aes}\NormalTok{(}\AttributeTok{x =}\NormalTok{ Attack, }\AttributeTok{y =}\NormalTok{ Defense)) }\SpecialCharTok{+}
  \FunctionTok{facet\_wrap}\NormalTok{(}\SpecialCharTok{\textasciitilde{}}\NormalTok{Type}\FloatTok{.1}\NormalTok{,                   }
             \AttributeTok{nrow =} \DecValTok{3}\NormalTok{) }\SpecialCharTok{+}
    \FunctionTok{geom\_point}\NormalTok{(}\AttributeTok{data =}\NormalTok{ df2,             }
             \AttributeTok{color =} \StringTok{"gray"}\NormalTok{,}
             \AttributeTok{alpha =} \FloatTok{0.35}\NormalTok{,}
             \AttributeTok{size =} \FloatTok{0.50}\NormalTok{) }\SpecialCharTok{+}
  \FunctionTok{geom\_smooth}\NormalTok{(}\AttributeTok{method =} \StringTok{\textquotesingle{}lm\textquotesingle{}}\NormalTok{,             }
              \AttributeTok{se =} \ConstantTok{FALSE}\NormalTok{,               }
              \AttributeTok{size =} \FloatTok{0.75}\NormalTok{) }\SpecialCharTok{+}
  \FunctionTok{geom\_point}\NormalTok{(}\FunctionTok{aes}\NormalTok{(}\AttributeTok{color =}\NormalTok{ Type}\FloatTok{.1}\NormalTok{),  }
             \AttributeTok{size =} \FloatTok{0.65}\NormalTok{,) }\SpecialCharTok{+}
  \FunctionTok{theme\_bw}\NormalTok{() }\SpecialCharTok{+} 
  \FunctionTok{theme}\NormalTok{(}\AttributeTok{legend.position =} \StringTok{\textquotesingle{}none\textquotesingle{}}\NormalTok{,           }
        \AttributeTok{strip.background =} \FunctionTok{element\_blank}\NormalTok{())}
\end{Highlighting}
\end{Shaded}

\begin{verbatim}
## `geom_smooth()` using formula 'y ~ x'
\end{verbatim}

\includegraphics{tidyverse-tutorial_files/figure-latex/unnamed-chunk-22-1.pdf}

\begin{Shaded}
\begin{Highlighting}[]
\FunctionTok{ggsave}\NormalTok{(}\StringTok{"beautiful{-}plot.pdf"}\NormalTok{, }\AttributeTok{dpi =} \DecValTok{600}\NormalTok{)}
\end{Highlighting}
\end{Shaded}

\begin{verbatim}
## Saving 6.5 x 4.5 in image
## `geom_smooth()` using formula 'y ~ x'
\end{verbatim}

And that's it! Manipulating things in R takes some time to get used to
and finding the right options for your needs may require some Google
searching (StackOverflow has saved me many times).

\hypertarget{resources-1}{%
\subsubsection{Resources}\label{resources-1}}

Here are some resources that I've found useful over the years:

\href{https://colorpalettes.net/category/contrasting-color/}{Color
Palettes} - contrasting color palettes

\href{https://coolors.co/6a0f49-d741a7-7d83ff-007fff-2d898b-7cb518-12664f-ffa62b-59c3c3-ebebeb}{Coolors}
- create color palettes, shuffle around colors until you find something
that works

\href{https://www.rstudio.com/wp-content/uploads/2015/02/data-wrangling-cheatsheet.pdf}{dplyr
cheat sheet}

\href{https://www.data-imaginist.com/2021/say-goodbye-to-good-taste/}{ggfx}
- A little late for April Fool's but if you don't think your PI will
kill you, you can send them a plot with a funny filter

\href{https://www.rstudio.com/wp-content/uploads/2016/11/ggplot2-cheatsheet-2.1.pdf}{ggplot
cheat sheet} - classic ggplot cheat sheet, hopefully after this tutorial
it's helpful!

\href{https://www.r-graph-gallery.com}{R graph gallery} - awesome
plotting examples for almost all types (code included)

\href{https://www.tidyverse.org}{tidyverse website}

\href{https://blog.datawrapper.de/which-color-scale-to-use-in-data-vis/}{Which
color scale for dataviz?} - great blogpost about choosing color scales
for different types of vizualization.

The \#tidytuesday hashtag on Twitter (I'm serious!)\\

\end{document}
